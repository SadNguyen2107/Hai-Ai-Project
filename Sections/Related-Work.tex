\section{Related Work}
The development of Artificial Intelligence (AI) is opening up new opportunities for the
beekeeping industry. AI technology allows for efficient remote monitoring, collecting
continuous data on various aspects of bee colony life. AI algorithms are applied to analyze
and process data, providing accurate information about the health and activity of bee
colonies. This enables beekeepers to monitor their colonies 24/7 without direct
intervention, detecting and addressing problems such as diseases or other harmful
factors in a timely manner. Consequently, they can optimize beekeeping, improve product
yield and quality, and save time and costs in colony management. Although this technology
has been widely researched and applied in many countries worldwide, the application of AI
in beekeeping in Vietnam is still limited. Research related to the detection of anomalies
in beekeeping can be mentioned as follows:

Several systems introduced by
\citeauthor{schurischuster2016sensor} (\citeyear{schurischuster2016sensor}) \cite{schurischuster2016sensor},
\citeauthor{zacepins2016remote} (\citeyear{zacepins2016remote}) \cite{zacepins2016remote},
\citeauthor{antonio2017frequency} (\citeyear{antonio2017frequency}) \cite{antonio2017frequency},
\citeauthor{crawford2017automated} (\citeyear{crawford2017automated}) \cite{crawford2017automated}
have used a multi-sensor beehive monitoring system
called BeePi, including a Raspberry Pi computer, a miniature camera, 4 microphones connected
to a splitter, a solar panel, a temperature sensor, a battery, and a clock. In the research
of \citeauthor{kulyukin2018toward} \cite{kulyukin2018toward}, microphones were placed to collect sound samples of bees,
crickets, and ambient noise. The authors then used machine learning models on bee sound
datasets collected from different locations to train and classify the collected sounds.
The experimental results achieved high accuracy, so it is entirely possible to use sound
to monitor the hive status.

Besides the above research on bee monitoring, in 2019, research by Ruvinga and colleagues
used the MFCC feature extraction method along with a CNN network to predict queen bee loss
sounds with an accuracy rate of up to 99\% on the Arnia Ltd. dataset
(\citeurl{ruvinga2023identifying}) \cite{ruvinga2023identifying},
which shows that beehive sound analysis technology has been used as an effective tool for
early detection of problems related to the queen bee, especially the loss of the queen. One
of the reasons why sound analysis is an effective tool for early detection of problems
related to the queen bee is that the transformation and change of sound are obvious when
the queen bee of the bee colony has a problem, for example: When the queen bee dies or
leaves, the hive sound can change from a quiet and rhythmic state to abnormal sounds such
as long and repetitive buzzing of worker bees. This characteristic sound is a clear sign of
instability in the hive. Compared to other inspection methods such as temperature or
humidity sensors, although temperature or humidity sensors can detect changes in the bee
colony environment, they cannot provide specific information about the status of the queen
bee or the bee colony. Sound analysis will have clearer data on queen loss based on the
activity and behavior of worker bees.

Based on the analysis of research on detection monitoring in the beekeeping process,
and then providing analysis to warn of queen loss, the research team chooses to
analyze the sound collected from the beehive to issue an early warning of queen loss for
deployment. This method provides high accuracy and is easy to automate, helping beekeepers
monitor their bee colonies continuously and reduce risks. Compared to other monitoring
methods, sound analysis not only detects early queen loss but also helps protect the health
of the bee colony without direct intervention in the hive.

In essence, this text discusses the application of AI, particularly sound analysis, in
beekeeping to detect problems like queen bee loss. It highlights the advantages of using
AI, such as remote monitoring, early detection, and improved efficiency. The text also
provides examples of existing research and the benefits of using sound analysis over other
methods.
